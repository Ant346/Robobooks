% Файл P_preamble.tex  % преамбула
\usepackage[warn]{mathtext} %(до задания inputenc, fontenc, babel)
% русские буквы в формулах, с предупреждением
%---- основной язык -- русский
\usepackage[T2A]{fontenc}
\usepackage[cp1251]{inputenc}
\usepackage[english,russian]{babel}   % load Babel setup for English
                                      % and Russian languages;
                                      % the latter is the default.
% Поскольку опция russian стоит последней, основным языком документа будет russian.

%-- форматирование страницы
\textheight 187mm
\textwidth 130mm
\headheight13.6pt
\special{papersize=170mm,240mm}
\oddsidemargin -5.4mm
\evensidemargin -5.4mm
\topmargin -5.4mm

%--- стиль заполнения таблицы
\clubpenalty=400
\widowpenalty=400
\tolerance=500  %max=10000, default=200 (выбор между разрежением и переполнением).
\looseness=-1 %(можно удлинять страницу на одну строку)
\hfuzz=2.5pt % можно вылезти за край строки на 2.5

%--- полезные пакеты
\usepackage{amssymb,amsmath} % почти стандарт Латеха
\usepackage{xspace}
\usepackage{enumerate} % списки
\usepackage{booktabs}
\usepackage[dotinlabels]{titletoc}

%-- включение рисунков
\usepackage{graphicx}  % Пакет для включения рисунков
%\usepackage[dvips]{color} % есть цветные фотографии!
\graphicspath{{figs/}} % В этой директории хранятся все рисунки *.eps,
\usepackage[usenames,dvipsnames]{xcolor,colortbl}

%--- мой цвет для таблиц, и для ссылок на рисунки и формулы
\definecolor{lightcyan}{rgb}{0.88,1,1} % обычно: ( 0.88, 1, 1)
\ifbool{For_Internet}{\newcolumntype{g}{>{\columncolor{lightcyan}}c}}%
{\newcolumntype{g}{>{\columncolor{light-gray}}c}}
\newcommand*{\red}[1]{\textcolor[rgb]{1.00,0.00,0.00}{#1}}
\newcommand*{\blue}[1]{\textcolor[rgb]{0.00,0.00,1.00}{#1}}

\ifbool{For_Internet}%
{\renewcommand{\thetable}{\red{{\it\arabic{chapter}.\arabic{table}\,}\normalfont}}}%
{\renewcommand{\thetable}{{\it\arabic{chapter}.\arabic{table}\,}\normalfont}}

\ifbool{For_Internet}%
{\renewcommand{\thefigure}{{\blue{\bfseries{\arabic{chapter}.\arabic{figure}}\normalfont}}}}%
{\renewcommand{\thefigure}{{\bfseries{\arabic{chapter}.\arabic{figure}}\normalfont}}}

% Чтобы окрасить названия глав в оглавлении
 \ifbool{For_Internet}{
 \titlecontents{chapter}[1.8em] % distance to page margin
   {\vspace{3mm} \bfseries\color{blue}} % \sffamily
   {\contentslabel[\thecontentslabel. ]{1.5em}} % distance between 1. and Title of chapter
   {\hspace*{-2.3em}}{\color{blue}
   {\titlerule*[1pc]{}\contentspage}\color{blue}}[\vspace{0.5mm}]
 }{}

\definecolor{light-gray}{gray}{0.95}
\ifbool{For_Internet}{\newcommand*{\CR}{\rowcolor{lightcyan}}}%
{\newcommand*{\CR}{\rowcolor{light-gray}}}

%---- оформление ссылок в виде \label{fig:name} \fref{fig:name}
\usepackage[vario]{fancyref} % plain is also possible

\renewcommand*{\fancyrefdefaultspacing}{\fancyreftightspacing}

\frefformat{vario}{\fancyreffiglabelprefix}{\bfseries{#1}\normalfont }
\frefformat{vario}{\fancyreftablabelprefix}{\textit{#1}\normalfont}
\frefformat{vario}{\fancyrefenumlabelprefix}{\textrm{#1}\normalfont}
% Если это убрать, то пакет fancyref начинает вставлять много отсебятины.

%--- мои макросы
\newcommand*{\mb}[1]{\mbox{\boldmath$#1$}}  % жирный мат. курсив

\newcommand*{\BE}{\begin{equation}}        %
\newcommand*{\EN}{\end{equation}}          %

\newcommand*{\BEA}{\begin{subequations} \begin{eqnarray}} %
\newcommand*{\ENA}{\end{eqnarray} \end{subequations}}   %

\newcommand*{\hs}{\hspace*{\parindent}}    %
\newcommand*{\nn}{\nonumber}        %

\newcommand*{\1}{$_1$}
\newcommand*{\2}{$_2$}
\newcommand*{\3}{$_3$}
\newcommand*{\4}{$_4$}
\newcommand*{\5}{$_5$}
\newcommand*{\6}{$_6$}
\newcommand*{\7}{$_7$}
\newcommand*{\8}{$_8$}
\newcommand*{\9}{$_9$}
%
\newcommand*{\T}[1]{\text{#1}}
%
\newcommand*{\vep}[3]{$(#1 \pm #2) \times \! 10 ^{#3} $\xspace} % (v+-e)*10^n
%
\newcommand*{\ic}{см$^{-1}$\xspace}                 % inverse centimeters

%----  Переопределение математических символов в русских традициях
\renewcommand{\le}{\leqslant}
\renewcommand{\leq}{\leqslant}
\renewcommand{\ge}{\geqslant}
\renewcommand{\geq}{\geqslant}

%---- вставить простую таблицу
\newcommand*{\TableBE}[5]{
\begin{table}[#1] %\captionabove
\vspace*{-#5mm}
\centering \sffamily \caption{\label{tab:#2}#3} \begin{tabular}{#4} \toprule }

\newcommand*{\TableEN}[3]{
\bottomrule \end{tabular}
\vspace{-#2mm}  \small \begin{flushleft}  #1 \end{flushleft}
\vspace{-#3mm}
\end{table}}

%---- вставить в текст таблицу
%  начало
\newcommand*{\WrapTableBE}[6]{
\renewcommand{\baselinestretch}{0.75}\small\normalsize
\begin{wraptable}{#1}{#5\textwidth} \sffamily
\begin{center} \vspace*{-#6mm}
\caption{\label{tab:#2}#3} \begin{tabular}{#4} \toprule }
% #1 : l, r   #2 : label   #3 : caption  #4 : cc|cc|rr|ll  #5 : 40mm

% вставить в текст таблицу, окончание
\newcommand*{\WrapTableEN}[2]{
\bottomrule \end{tabular} \end{center} \vspace{-#2pt} \small #1  \end{wraptable}
\renewcommand{\baselinestretch}{1}\small\normalsize}

%---- вставить рисунок внутрь текста
\usepackage{wrapfig} % нужен для RisEpsRight

\newcommand*{\EpsWrapD}[7]{%
\begin{wrapfigure}[#5]{#3}{#2 \textwidth} %
\begin{center} \sffamily
\includegraphics*[width= #2 \textwidth ]{#1}
\vspace{-#7mm}
\caption{\label{fig:#1}#4}
\vspace{-#6pt}
\end{center}
\end{wrapfigure}}

%---- потоки в другие файлы
\usepackage{clipboard} % It is used to |copy and \HL\Paste the texts with tasks
\usepackage{newfile}
\newoutputstream{ZO}
\openoutputfile{Zadachi.tex}{ZO}
\newcommand*{\itemZO}[2]{\item  \label{enum:#1}%
\Copy{#1}{\emph{\ifbool{For_Internet}{\blue{ #2 }}{#2}}}%
\addtostream{ZO}{$\{$\thechapter.\theenumi $\}$}}

\endinput
