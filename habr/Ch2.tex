%+++++++++++++++++
% Файл \Ch2.tex  % глава 2
\chapter{\label{ch:Chast2}Про букву ё и кавычки}
\section{Передохнём или передохнем?}
\hs
   Для тех, кто протестует против ничем не заслуженной дискриминации буквы \emph{ё},
есть специальная программа Yo.
   Скачайте её и вы сможете ёфицировать свой текст!
   Долой дискриминацию буквы \emph{ё}!

\section{Русские кавычки}
\hs
   В английском языке приняты одинарные и двойные кавычки в виде ‘...’ и “...”.
   В России приняты французские («...») и немецкие („...“) кавычки, они называются «ёлочки» и «лапки» соответственно.

\WrapTableBE{r}{C2_parties}{Ответы политпартий на вопросы}{g llll}{0.3}{8}
\CR   & I  & II  & III  & IV   \\
\midrule
ЕдРо  & 0 & 0 & 1 & 1  \\
ЛДПР  & 1 & 1 & 0 & 1  \\
КПРФ  & 0 & 1 & 1 & 0  \\
\WrapTableEN{I, II, III, IV -- номера вопросов, 0 и 1 -- ответы (да, нет)}{5} %
%
   «Лапки» используются внутри «ёлочек» («Что тебе снится, крейсер „Аврора“»).

   Для набора «лапок» можно использовать команды \verb|\glqq|  и \verb|\grqq|,
а для «ёлочек» --– \verb|\flqq|  и \verb|\frqq|.
   Они определены в пакете babel.

   Но это не обязательно.
   Например, «лапок» на клавиатуре нет, а скопировать их откуда-нибудь можно и компилятор их понимает.

\section{Задачи к главе \thechapter}
\begin{enumerate}[$\{$\thechapter.1$\}$]
\begin{writeverbatim}{ZO}
\section{Задачи и решения к главе~\ref{ch:Chast2}}
\end{writeverbatim}
%---------------------
\itemZO{Zadacha21}{В табл.~\fref{tab:C2_parties} показаны ответы на важные вопросы трёх политических партий.
   По этим ответам можно вычислить коэффициенты антикорреляции (КА) между любыми двумя партиями, $0 \le$КА$\le 1$.
   Всегда ли можно нанести эти партии на плоскость в виде трёх точек, чтобы расстояние между любыми двумя партиями было пропорционально КА между ними? }
\begin{writeverbatim}{ZO}
\Paste{Zadacha21}\hs

\EpsWrapD{P_ABC}{0.28}{r}{Три партии}{8}{0}{6}

   Ответ.
   Для построения треугольника нужно, чтобы выполнялось неравенство треугольника $a+b \ge c$, где
($a,b,c$) -- любая выборка из трёх длин, см. рис. \fref{fig:P_ABC}.

   Представим себе крайний случай -- у одной партии все нули, а у другой все единицы.
   Расстояние между ними равно 1. Тогда у третьей партии сумма коэффициентов будет тоже равна 1, то есть она располагается на отрезке, соединяющем первые две партии.
   Все остальные ситуации проще.

\end{writeverbatim}
\end{enumerate}
\endinput
% Программа (TikZ) для этого рисунка:
\tikzsetnextfilename{P_ABC}
\begin{tikzpicture}
\coordinate(a) at (1,0);
\coordinate(b) at (-1.3,0);
\coordinate(c) at (0,1);
\draw[very thick] (a) --node[below]{$a$} (b)--node[above left]{$b$} (c)--node[above right]{$c$} (a);
\shade[ball color=blue] (a) circle(0.3) node[below=3mm]{ЛДПР};
\shade[ball color=red]  (b) circle(0.4) node[below=4mm]{КПРФ};
\shade[ball color=green](c) circle(0.5) node[above=5mm]{ЕдРо};
\end{tikzpicture}