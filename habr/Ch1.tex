%+++++++++++++++++
% Файл \Ch1.tex  % глава 1
\chapter{\label{ch:Chast1}Про сжатие текста}
\section{Маленькие рисунки}
\hs
   Латех охотно отдаёт под плавающий объект (рисунок, таблицу) целые строчки, но очень не любит обтекание плавающих объектов текстом.
   Печальный факт состоит в том, что вписать в текст рисунок с подписью
снизу или сверху в Латехе до сих пор проблема.
   Это делают пакеты wrapfig и floatflt, но оба капризны.
   Так что выбор есть, но он плохой.

   Если вы видите маленький красивый векторный рисунок и
подпись к нему расположена слева или справа,
так что рисунок и подпись занимают всю ширину страницы,
то знайте -- это фирменный стиль пакета TikZ.
%   Люблю его, как тёщу.  То есть совсем не очень.
   Одна из его поганых особенностей состоит в том, что получение pdf-файла
по схеме tex$\to$dvi$\to$ps$\to$pdf очень сильно отличается
от результатов, полученных по схеме tex$\to$dvi$\to$pdf.

\section{Как форматировать текст}
\hs
   Представим себе, что вам нужно ужать текст.
   Например, чтобы убрать висящую в начале страницы строку, после которой начинается новая глава.

   Пример внизу показывает один из полезных трюков -- разное размещение индексов в знаке суммы.
   Заодно показано введение текста в формулу в математической и в текстовой моде.
\BEA
   \sum_{\varkappa=0,...}(1/2)^{\varkappa} \leq {\mb 2}, & \T{скобка справа от text, $e^x$, \vep{2}{1}{3}   },\\
   \sum\nolimits_{\phi=0,...}(1/2)^{\phi} \le {\mb 2},   & {\text скобка слева от text, \tg(x) }.
\ENA

\section{Задачи к главе \thechapter}
\begin{enumerate}[$\{$\thechapter.1$\}$]
\begin{writeverbatim}{ZO}
\section{Задачи и решения к главе~\ref{ch:Chast1}}
\end{writeverbatim}
%---------------------
\itemZO{Zadacha11}{Докажите, как физик физику, что $f= \int_0^{\infty} \cos(x/a) \; dx =0$.}
\begin{writeverbatim}{ZO}
\Paste{Zadacha11}\smallskip\\
\hs
   Доказательство.
   Пусть математики думают, что $f = \lim_{R \to \infty} \sin(R/a) = ??$, это их дело.
   А мы добавим какую-нибудь физическую причину, меняющую определение:
\BE \nn
f = \int_0^{\infty} e^{-bx}\; \cos(x/a) \; dx = \frac{b}{b^2+1/a^2}.
\EN
   Теперь надо перейти к частным случаям.
   Например, пренебречь физической причиной, положив  $b=0$.
   Получим $f = 0$, что и требовалось доказать.

\end{writeverbatim}
\end{enumerate}