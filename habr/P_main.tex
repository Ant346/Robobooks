% Файл P_main.tex  % главный в проекте
\documentclass[10pt,twoside,openany]{book}
\usepackage{etoolbox}   % this package introduces operations: \newbool,...

\newbool{For_Internet}
\booltrue{For_Internet}    % % Цветная книга для Интерента
%\boolfalse{For_Internet}  % или черно-белая для бумаги

\newbool{FR} %FR= First Run
%\booltrue{FR}  %  --  make  FR = true
\boolfalse{FR}  %  --  make  FR = false

\input{P_Preamble}       % загрузить преамбулу
\includeonly{             % какие главы включаем в книгу
P_Ch1,        %%%
P_Ch2,        %%%
Zadachi_Head  %%%
}
\begin{document}         % начало документа
\tableofcontents         % Оглавление

\ifbool{FR}{\newclipboard{myclipboard}}{}
                          % какие главы компилируем
\include{P_Ch1}          %%%
\include{P_Ch2}          %%%

\closeoutputstream{ZO}  % закрыть поток в файл Zadachi.tex
\ifbool{FR}{}{\openclipboard{myclipboard}}

% Файл Zadachi_Head.tex % Заголовок части с задачами и решениями
\chapter{Решения задач}
\input{Zadachi}
\endinput % загрузить Файл с задачами с решениями
%\bibliography{P_Bibs}  % создать список литературы, убираем простоты ради

\end{document}         % конец документа
\endinput              % конец текста, дальше только комментарии
%+++++++++++++++++