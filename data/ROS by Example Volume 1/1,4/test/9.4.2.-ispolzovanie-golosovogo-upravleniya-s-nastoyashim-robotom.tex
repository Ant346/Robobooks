# 9.4.2. Использование голосового управления с настоящим роботом


Для голосового управления TurtleBot переместите робота в открытое пространство без препятствий, затем запустите файл turtlebot.launch на ноутбуке TurtleBot:

```text
$ roslaunch rbx1_bringup turtlebot_minimal_create.launch
```

Если он еще не запущен, вызовите rqt\_console, чтобы упростить мониторинг вывода скрипта голосовой навигации:

```text
$ rqt_console &
```

Перед запуском сценария голосовой навигации проверьте настройки звука, как описано ранее, чтобы убедиться, что ваш микрофон по-прежнему установлен в качестве устройства ввода.

На вашем компьютере рабочей станции запустите файл voice\_nav\_commands.launch:

```text
$ roslaunch rbx1_speech voice_nav_commands.launch
```

и в другом терминале запустите файл turtlebot\_voice\_nav.launch:

```text
$ roslaunch rbx1_speech turtlebot_voice_nav.launch
```

Сначала попробуйте относительно безопасную голосовую команду, например, «повернуть вправо». Обратитесь к списку команд выше для различных способов перемещения робота. Файл turtlebot\_voice\_nav.launch содержит параметры, которые вы можете установить, которые определяют максимальную скорость TurtleBot, а также приращения, используемые, когда вы говорите «идти быстрее» или «замедляться».

В следующем видео представлена ​​короткая демонстрация скрипта, работающего на модифицированном TurtleBot: \href{http://www.youtube.com/watch?v=10ysYZUX_jA}{http://www.youtube.com/watch?v=10ysYZUX\_jA}

