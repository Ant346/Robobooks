\Subsection{4.4 Создание catkin Workspace}


Если вы еще не сделали этого, создайте каталог для хранения рабочей области catkin. Условно мы будем считать, что вы создали подкаталог под названием catkin\_ws в своем домашнем каталоге. Ниже этого каталога мы создаем подкаталог под названием\textit{src}для хранения исходных файлов пакета. Если вы начинаете с нуля, выполните следующие команды, чтобы создать пустое рабочее пространство catkin в каталоге ~ / catkin\_ws:

```text
$ mkdir -p ~/catkin_ws/src 
$ cd ~/catkin_ws/src
$ catkin_init_workspace
```

Первая команда выше создает как каталог верхнего уровня ~/catkin\_ws, так и подкаталог src. Обратите также внимание, что мы запускаем команду catkin\_init\_workspace в каталоге src. 

Далее, Даже если текущая рабочая область пуста, мы запускаем catkin\_make, чтобы создать некоторые начальные каталоги и установочные файлы. catkin\_make всегда выполняется в папке рабочего пространства верхнего уровня catkin\_ws (не в папке src):

```text
$ cd ~/catkin_ws
$ catkin_make
```

\textbf{Примечание 1}: после создания любого нового пакета(ов) catkin, убедитесь, что источник ~/catkin\_ws/devel/setup.bash файл и перестроить путь к пакету ROS следующим образом:

```text
$ source ~/catkin_ws/devel/setup.bash 
$ rospack profile
```

Это гарантирует, что ROS сможет найти любые новые пакеты, типы сообщений и модули Python, принадлежащие вновь созданным пакетам.

\textbf{Примечание 2}: Добавьте исходную команду выше в конец вашего ~/.файл bashrc, чтобы новые терминалы автоматически забирали ваши пакеты catkin. Либо отредактируйте свой ~/.bashrc файл вручную или выполните команду:

```text
$ echo "source ~/catkin_ws/devel/setup.bash" >> ~/.bashrc
```



