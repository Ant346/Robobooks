# 4.17.2 Использование команды roslocate


Если вы знаете точное имя пакета, которое вы ищете, и хотите найти URL-адрес хранилища пакетов, используйте команду roslocate. (Это обычно доступно, если вы установили rosinstall, как описано ранее.) Например, чтобы найти расположение пакета ros\_arduino\_bridge для ROS Groovy, выполните команду:

```text
$ roslocate uri ros_arduino_bridge
```

который должен дать результат:

> ROS by Example
>
> Using ROS\_DISTRO: indigo  
>  Not found via rosdistro - falling back to information provided by rosdoc https://github.com/hbrobotics/ros\_arduino\_bridge.git

Это означает, что мы можем установить пакет в наш личный каталог catkin с помощью команды git:

```text
$ cd ~/catkin_ws/src
$ git clone git://github.com/hbrobotics/ros_arduino_bridge.git $ cd ~/catkin_ws
$ catkin_make
$ source devel/setup.bash
$ rospack profile
```

\textbf{ПРИМЕЧАНИЕ}. Начиная с ROS Groovy, команда roslocate будет возвращать результат только в том случае, если пакет или стек были переданы в индексатор сопровождающим пакета для используемого вами ROS-дистрибутива.

