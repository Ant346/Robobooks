\Section{3. ОПЕРАЦИОННЫЕ СИСТЕМЫ И ВЕРСИИ ROS}


ROS может работать на различных версиях Linux, MacOS X и частично на Microsoft Windows. Однако самый простой способ начать работу - это использовать Ubuntu Linux, поскольку эта ОС официально поддерживается OSRF. Кроме того, Ubuntu бесплатна и проста в установке вместе с другими операционными системами. Если вы еще не используете Ubuntu, мы перечислим несколько советов по его установке в следующем разделе.

ROS может работать на любом устройстве, от суперкомпьютера до Beagleboard. Но большая часть кода, который мы разработаем, потребует некоторой интенсивной обработки процессора. Поэтому вы сэкономите немного времени и потенциальных разочарований, если будете использовать ноутбук или другой полноразмерный ПК для запуска примеров приложений из этой книги. Turtlebot, Maxwell и Pi Robot разработаны таким образом, чтобы вы могли легко разместить ноутбук на роботе. Это означает, что вы можете разработать свой код прямо на ноутбуке, а затем просто вставить его в робота для тестирования автономного поведения. Вы даже можете пойти наполовину во время разработки и привязать робота к ноутбуку или ПК, используя несколько длинных USB-кабелей.

Эта версия книги была протестирована на ROS Indigo и Ubuntu 14.04 (Trusty), которая является текущей версией Ubuntu для долгосрочной поддержки (LTS).

\textbf{ПРИМЕЧАНИЕ}. Обычно требуется, чтобы одна и та же версия ROS использовалась на всех компьютерах, подключенных к одной и той же сети ROS. Это включает в себя компьютер на вашем роботе и настольную рабочую станцию. Причина в том, что сигнатуры сообщений ROS, как известно, изменяются от одного выпуска к другому (например, диагностическое сообщение MD5 изменяется между Hydro и Indigo), так что разные выпуски, как правило, не могут использовать темы и службы друг друга.

