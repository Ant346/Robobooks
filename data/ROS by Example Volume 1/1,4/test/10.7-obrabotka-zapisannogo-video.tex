# 10.7 Обработка записанного видео


Пакет rbx1\_vision также содержит узел video2ros.py для преобразования записанных видеофайлов в видеопоток ROS, чтобы вы могли использовать его вместо живой камеры. Чтобы проверить узел, прекратите работу любых драйверов камер, которые у вас могут работать в другом терминале. Также завершите работу узла ros2opencv2.py, если он все еще работает. Затем выполните следующие команды:

```text
$ rosrun rbx1_vision cv_bridge_demo.py
```

> \[INFO\] \[WallTime: 1362334257.368930\] Waiting for image topics...

```text
$ roslaunch rbx1_vision video2ros.launch input:=`rospack find \ rbx1_vision`/videos/hide2.mp4
```

(Тестовое видео предоставлено видео-базой данных \href{http://vision.ucsd.edu/~leekc/HondaUCSDVideoDatabase/HondaUCSD.html}{Honda / UCSD.})

Вы должны увидеть два активных окна отображения видео. (Окно глубины видео останется пустым.) Окно отображения, называемое «Воспроизведение видео», позволяет вам контролировать записанное видео: щелкните в любом месте окна, чтобы вывести его на передний план, затем нажмите пробел, чтобы приостановить / продолжить видео и нажмите клавишу «r», чтобы перезапустить видео с самого начала. Другое окно отображает выходные данные нашего узла cv\_bridge\_demo.py, который, как вы помните, вычисляет карту границ входных данных.

Скрипт video2ros.py комментируется и не требует пояснений. Вы можете найти источник в Интернете по следующей ссылке:

Ссылка на источник: \href{https://github.com/pirobot/rbx1/blob/indigo-devel/rbx1_vision/nodes/video2ros.py}{video2ros.py}

Теперь, когда наши основные узлы видения работают, мы готовы опробовать ряд Функции обработки зрения OpenCV.

