# 4.18 Получение дальнейшей помощи с ROS


Есть несколько источников для дополнительной помощи с ROS. Вероятно, лучшее место для начала - это главная вики ROS по адресу \href{http://wiki.ros.org}{http://wiki.ros.org}. Как описано в предыдущем разделе, обязательно используйте поле поиска в правом верхнем углу страницы.

 Если вы не можете найти то, что вы ищете в вики, попробуйте форум вопросов и ответов ROS по адресу \href{http://answers.ros.org}{http://answers.ros.org}. Сайт ответов - отличное место, чтобы получить помощь. Вы можете просматривать список вопросов, выполнять поиск по ключевым словам, искать темы на основе тегов и даже получать уведомления по электронной почте при обновлении темы. Но не забудьте выполнить какой-либо поиск, прежде чем публиковать новый вопрос, чтобы избежать дублирования.

Далее вы можете найти один из архивов списка рассылки ROS: 

\begin{itemize} 
\item { \href{http://code.ros.org/lurker/list/ros-users.html}{ros-users}: для общих новостей и объявлений ROS} 
\item { \href{http://kinect-with-ros.976505.n3.nabble.com/}{ros-kinect}: для вопросов, связанных с Kinect} 
\item { \href{http://www.pcl-users.org/}{pcl-users}: для вопросов, связанных с PCL} 
\end{itemize} 

\textbf{ПРИМЕЧАНИЕ}. Пожалуйста, не используйте список рассылки ros-users для публикации вопросов об использовании ROS или отладочные пакеты. Вместо этого используйте \href{http://answers.ros.org}{http://answers.ros.org}. 

Если вы хотите подписаться на один или несколько из этих списков, используйте соответствующую ссылку в списке ниже:

\begin{itemize} 
\item { \href{http://lists.ros.org/mailman/listinfo/ros-users}{ros-users}: страница подписки } 
\item {  ros-kinect - возможно, этот список больше не активен.} 
\item {  \href{%20http://pointclouds.org/mailman/listinfo/pcl-users}{pcl\_users} - страница подписки} 
\end{itemize} 



