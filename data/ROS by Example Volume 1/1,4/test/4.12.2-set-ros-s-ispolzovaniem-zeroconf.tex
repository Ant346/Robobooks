# 4.12.2 Сеть ROS с использованием Zeroconf


Более поздние версии Ubuntu включают поддержку \href{http://en.wikipedia.org/wiki/Zero_configuration_networking}{Zeroconf}, метода, который позволяет машинам в одной подсети ссылаться друг на друга, используя локальные имена хостов вместо IP-адресов. Таким образом, вы можете использовать этот метод, если ваш настольный компьютер и робот подключены к одному и тому же маршрутизатору в домашней или офисной сети—общий сценарий как для хобби, так и для исследовательских роботов. 

Используйте команду hostname для определения короткого имени Вашего компьютера. Результатом будет имя, которое вы выбрали во время первоначальной установки Ubuntu на этом компьютере. Например, если вы назвали свой настольный компьютер "my\_desktop", выходные данные будут выглядеть следующим образом:

```text
$ hostname my_desktop
```

Чтобы получить имя хоста Zeroconf, просто добавьте ".local " после имени хоста, так что в этом случае имя хоста Zeroconf будет:

> my\_desktop.local

Затем выполните команду hostname на компьютере вашего робота, чтобы получить его имя хоста и добавить ".local", чтобы получить свое имя Zeroconf. Давайте предположим, что имя вашего робота Zeroconf-это:

> my\_robot.local

Теперь нам нужно проверить, могут ли машины видеть друг друга в сети.

