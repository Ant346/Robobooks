\Subsection{4.2 Установка rosinstall}


Утилита rosinstall не является частью установки рабочего стола ROS, поэтому нам нужно получить ее отдельно. Если вы будете следовать полному набору инструкций на странице установки \href{http://wiki.ros.org/indigo/Installation/Ubuntu%23Getting_rosinstall}{Indigo}, вы уже выполнили эти шаги. Однако многие пользователи, как правило, пропускают эти последние шаги:  


```text
$ sudo apt-get install python-rosinstall 
$ sudo rosdep init
$ rosdep update
```

  
  
Последняя команда запускается как обычный пользователь, то есть без sudo.

\textbf{ПРИМЕЧАНИЕ}. Если вы использовали более ранние версии ROS на той же машине, на которой установлен Indigo, могут возникнуть проблемы с запуском нового пакета python-rosinstall вместе с более старыми версиями rosinstall, которые могли быть установлены с помощью pip или easy\_install. Подробнее о том, как удалить эти более старые версии, см. В этом ответе на \href{http://answers.ros.org/question/44186/how-do-i-solve-fuerte-overlay-creation-problem/}{answers.ros.org.}

