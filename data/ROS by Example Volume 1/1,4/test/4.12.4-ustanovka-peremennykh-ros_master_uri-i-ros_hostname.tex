# 4.12.4 Установка переменных ROS\_MASTER\_URI и ROS\_HOSTNAME




В любой дорожной сети одна машина назначается мастером ROS, и только она управляет процессом roscoreprocess. Затем другие машины должны установить переменную окружения theROS\_MASTER\_URI так, чтобы она указывала на главный хост. Каждый компьютер также должен установить свое имя хоста ROS соответствующим образом, как мы покажем.

 В общем, не имеет значения, какую машину вы выберете в качестве мастера. Однако для полностью автономного робота вы, вероятно, захотите сделать компьютер робота хозяином, чтобы он никоим образом не зависел от рабочего стола. 

Если мы хотим, чтобы робот был мастером ROS, мы устанавливаем его ROS\_HOSTNAME на его имя Zeroconf и запускаем процесс roscore:

_\textbf{На роботе:}_

```text
$ export ROS_HOSTNAME=my_robot.local
$ roscore
```

Затем перейдите на рабочий стол, установите ROS\_HOSTNAME на его имя Zeroconf, а затем установите переменная ROS\_MASTER\_URI указывает на URI Zeroconf вашего робота.

_\textbf{На рабочем столе:}_

```text
$ export ROS_HOSTNAME=my_desktop.local
$ export ROS_MASTER_URI=http://my_robot.local:11311
```

Для дополнительной проверки синхронизации времени мы можем запустить команду ntpdate для синхронизации рабочего стола с роботом.

_\textbf{На рабочем столе:}_

```text
$ sudo ntpdate -b my_robot.local
```

Если все пойдет хорошо, вы должны быть в состоянии увидеть темы /rosin и /rosin\_agg на ваш рабочий стол выглядит следующим образом:

```text
$ rostopic list
```

> ```text
> /rosout
> /rosout_agg
> ```

