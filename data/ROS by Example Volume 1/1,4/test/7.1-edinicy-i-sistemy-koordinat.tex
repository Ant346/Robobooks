\Subsection{7.1 Единицы и системы координат}


Прежде чем мы сможем отправить команды движения нашему роботу, нам нужно Просмотрите единицы измерения и условные обозначения системы координат, используемые в ROS. 

При работе с опорными кадрами, имейте в виду, что ROS использует соглашение правой руки для ориентации координатных осей, как показано слева. Индекс и середина пальцы указывают вдоль положительных осей X и Y и большой палец указывает в направлении положительной оси Z. Направление вращения вокруг оси определяется правилом правой руки, показанным справа: если вы направите большой палец в положительном направлении любой оси, ваши пальцы скручиваются в направлении положительного вращения. Для мобильного робота, использующего ROS, ось x указывает вперед, ось y указывает влево, а ось z вверх. Согласно правилу правой руки, положительное вращение робота вокруг оси z происходит против часовой стрелки, а отрицательное вращение по часовой стрелке. 

Помните также, что ROS использует метрическую систему, так что линейные скорости всегда указываются в метрах в секунду (м / с), а угловые скорости - в радианах в секунду (рад / с). Линейная скорость 0,5 м / с на самом деле довольно высокая для внутреннего робота (около 1,1 миль в час), в то время как угловая скорость 1,0 рад / с эквивалентна примерно одному обороту за 6 секунд или 10 об / мин. Если есть сомнения, начинайте медленно и постепенно увеличивайте скорость. Для внутреннего робота я склонен поддерживать максимальную линейную скорость на уровне 0,2 м / с или ниже.

