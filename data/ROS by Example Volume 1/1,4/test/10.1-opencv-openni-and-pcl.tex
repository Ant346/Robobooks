# 10.1 OpenCV, OpenNI and PCL




Три столпа компьютерного зрения в сообществе ROS - это \href{http://opencv.org/ё}{OpenCV}, \href{http://structure.io/openni}{OpenNI2}_,_ \href{https://github.com/OpenKinect}{+ OpenKinect} и \href{http://pointclouds.org/}{PCL}. OpenCV используется для обработки 2D-изображений и машинного обучения. OpenNI2 и OpenKinect предоставляют драйверы для глубинных камер, таких как Microsoft Kinect и Asus Xtion Pro. И PCL, или Библиотека облаков точек, является библиотекой выбора для обработки трехмерных облаков точек. В этой книге мы сосредоточимся на OpenCV, но также предоставим краткое введение в OpenNI / OpenKinect и PCL. (Для тех читателей, которые уже знакомы с OpenCV и PCL, вам также может быть интересна \href{http://plasmodic.github.com/ecto/}{Ecto}, концепция видения от Willow Garage, которая позволяет получить доступ к обеим библиотекам через общий интерфейс.)

В этой главе мы научимся:

\begin{itemize} 
\item { подключиться к веб-камере или RGB-D (глубине) камеры с помощью ROS } 
\item {  использовать утилиту ROS cv\_bridge для обработки потоков изображений ROS с помощью OpenCV } 
\item { писать программы ROS для обнаружения лиц, отслеживания ключевых точек с помощью оптического потока и отслеживания объектов определенного цвета •} 
\item { отслеживать скелет пользователя с помощью камеры RGB-D и OpenNI / OpenKinect } 
\item { определить ближайшего человека с помощью PCL} 
\end{itemize} 

