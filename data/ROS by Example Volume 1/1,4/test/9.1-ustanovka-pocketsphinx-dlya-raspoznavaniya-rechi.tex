\Subsection{9.1 Установка PocketSphinx для распознавания речи}


Благодаря Майклу Фергюсону мы можем использовать пакет ROS \href{http://wiki.ros.org/pocketsphinx}{pocketsphinx} для распознавания речи. Пакет pocketsphinx требует установки пакета Ubuntu gstreamer0.10-pocketsphinx, и нам также потребуется стек звуковых драйверов ROS (на случай, если у вас его еще нет), поэтому давайте сначала позаботимся об обоих. Вам будет предложено установить пакеты Festival, если у вас их еще нет - ответьте «Y», если будет предложено:

```text
$ sudo apt-get install gstreamer0.10-pocketsphinx 
$ sudo apt-get install ros-indigo-pocketsphinx
$ sudo apt-get install ros-indigo-audio-common
$ sudo apt-get install libasound2
```

В пакет pocketsphinx входит узелognizer.py. Этот скрипт выполняет всю тяжелую работу по подключению к аудиовходу вашего компьютера и сопоставлению голосовых команд со словами или фразами в текущем словаре. Когда узел распознавателя соответствует слову или фразе, он публикует его в теме /ognizer / output. Другие узлы могут подписаться на эту тему, чтобы узнать, что только что сказал пользователь.

