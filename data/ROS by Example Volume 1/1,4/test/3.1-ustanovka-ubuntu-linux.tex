\Subsection{3.1 Установка Ubuntu Linux}


Если у вас уже есть Ubuntu Linux на вашем компьютере, отлично. Но если вы начинаете с нуля, установить Ubuntu не сложно.

Эта версия книги была протестирована на ROS Indigo и Ubuntu 14.04 (Trusty), которая является текущей версией Ubuntu для долгосрочной поддержки (LTS). Список версий Ubuntu, которые официально совместимы с ROS Indigo, см. \href{http://wiki.ros.org/indigo/Installation/Ubuntu}{В Руководстве по установке ROS.}

Ubuntu можно установить на существующий компьютер с Windows или Mac на базе Intel, и вы, вероятно, захотите оставить эти установки без изменений при установке Ubuntu вместе с ними. После этого у вас будет выбор операционных систем для использования во время загрузки. С другой стороны, если у вас есть запасной ноутбук, который вы можете посвятить своему роботу, вы можете установить Ubuntu на весь диск. (Он также будет иметь тенденцию работать быстрее таким образом). В целом, не стоит устанавливать Ubuntu внутри виртуальной машины, такой как VMware. Хотя это отличный способ пробежаться по учебникам ROS, виртуальная машина, скорее всего, застрянет при попытке запустить программы с интенсивной графикой, такие как RViz. (На самом деле, такие программы могут вообще не работать).

Страница загрузки Ubuntu содержит \href{http://www.ubuntu.com/download/desktop}{ссылки} для установки Ubuntu с USB-накопителя или DVD-диска для Windows или MacOS X.

Чтобы эта книга была сфокусирована на самой ROS, пожалуйста, используйте Google или форумы поддержки Ubuntu, если у вас возникли проблемы с установкой.

