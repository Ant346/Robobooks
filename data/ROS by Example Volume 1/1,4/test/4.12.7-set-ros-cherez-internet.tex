# 4.12.7 Сеть ROS через Интернет


Хотя это и не входит в рамки данной книги, настройка узлов ROS для связи через Интернет аналогична приведенным выше инструкциям с использованием Zeroconf. Главное отличие заключается в том, что теперь вам нужно использовать полные имена хостов или IP-адреса вместо локальных имен Zeroconf. Кроме того, вполне вероятно, что одна или несколько машин будут находиться за брандмауэром, так что придется настроить некоторую форму VPN (например, \href{https://help.ubuntu.com/14.04/serverguide/openvpn.html}{OpenVPN}). Наконец, поскольку большинство машин в сети ROS будет подключено к локальному маршрутизатору (напр. точка доступа Wi-Fi), вам нужно будет настроить переадресацию портов на этом маршрутизаторе или использовать динамический DNS. Хотя все это возможно, это определенно не тривиальная установка.

