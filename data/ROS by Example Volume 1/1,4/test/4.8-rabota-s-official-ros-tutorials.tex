\Subsection{4.8 Работа с Official ROS Tutorials}


Прежде чем погрузиться в учебники для начинающих на Ros Wiki, настоятельно рекомендуется сначала ознакомиться с руководством \href{http://wiki.ros.org/ROS/StartGuide}{ROS Start Guide}. Здесь вы найдете общее введение, а также объяснение ключевых концепций, библиотек и технического дизайна. 

Официальные \href{http://wiki.ros.org/ROS/Tutorials}{учебники} для начинающих ROS превосходно написаны и были протестированы многими новыми пользователями ROS. Поэтому очень важно начать свое знакомство с ROS, последовательно прогоняя эти учебные пособия. Убедитесь, что вы действительно запускаете образцы кода—а не просто читаете их. Сам по себе ROS не является сложным: хотя на первый взгляд он может показаться незнакомым, чем больше вы практикуетесь с кодом, тем легче он становится. Ожидайте, что вы потратите по крайней мере несколько дней или недель на изучение учебных пособий. 

После того, как вы закончили учебники для начинающих, Важно также прочитать обзор \href{http://wiki.ros.org/tf}{tf} overview и сделать учебники \href{http://wiki.ros.org/tf/Tutorials}{tf} Tutorials, которые помогут вам понять, как ROS обрабатывает различные системы отсчета. Например, местоположение объекта в изображении камеры обычно задается относительно кадра, прикрепленного к камере, но что делать, если вам нужно знать местоположение объекта относительно основания робота? Библиотека Ros tf выполняет большую часть тяжелой работы для нас, и такие преобразования фреймов становятся относительно безболезненными для выполнения. 

Наконец, нам нужно будет понять основы действий \href{http://wiki.ros.org/actionlib}{ROS,} когда мы доберемся до навигационного стека. Вы можете найти введение в действия ROS в учебниках \href{http://wiki.ros.org/actionlib/Tutorials}{actionlib.} 

Вы заметите, что большинство учебников содержат примеры как на Python, так и на C++. Весь код в этой книге написан на Python, но если вы программист на C++, не стесняйтесь делать уроки на C++, так как концепции те же самые. 

Многие отдельные пакеты ROS имеют свои собственные учебные пособия на Вики-страницах ROS. Мы будем ссылаться на них по мере необходимости на протяжении всей книги, но на данный момент достаточно учебников \href{http://wiki.ros.org/ROS/Tutorials}{Beginner}, \href{http://wiki.ros.org/tf/Tutorials}{tf} и \href{http://wiki.ros.org/actionlib/Tutorials}{actionlib}.

