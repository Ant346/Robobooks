# Table of contents


\begin{itemize} 
\item { \href{README.md}{Титульный лист}} 
\item { \href{1.-smysl-dannoi-knigi.md}{1. СМЫСЛ ДАННОЙ КНИГИ}} 
\item { \href{2.-realnye-i-smodelirovannye-roboty.md}{2. РЕАЛЬНЫЕ И СМОДЕЛИРОВАННЫЕ РОБОТЫ}} 
\item { \href{2.1-gazebo-etapy-i-simulyator-arbotix.md}{2.1 Gazebo, этапы и симулятор ArbotiX}} 
\item { \href{2.2-znakomstvo-s-robotami-turtlebot-maxwell-i-pi-robot.md}{2.2 Знакомство с роботами TurtleBot, Maxwell и Pi Robot}} 
\item { \href{3.-operacionnye-sistemy-i-versii-ros.md}{3. ОПЕРАЦИОННЫЕ СИСТЕМЫ И ВЕРСИИ ROS}} 
\item { \href{3.1-ustanovka-ubuntu-linux.md}{3.1 Установка Ubuntu Linux}} 
\item { \href{3.2-nachalo-raboty-s-linux.md}{3.2 Начало работы с Linux}} 
\item { \href{3.3-primechanie-ob-obnovleniyakh-i-izmeneniyakh.md}{3.3 Примечание об обновлениях и изменениях}} 
\item { \href{4.1-ustanovka-ros.md}{4. ОБЗОР ОСНОВ ROS}} 
\item { \href{4.2-ustanovka-rosinstall.md}{4.2 Установка rosinstall}} 
\item { \href{4.3-postroenie-paketov-ros-s-pomoshyu-catkin.md}{4.3 Построение пакетов ROS с помощью Catkin}} 
\item { \href{4.4-sozdanie-catkin-workspace.md}{4.4 Создание catkin Workspace}} 
\item { \href{4.5-vypolnenie-make-clean-s-pomoshyu-catkin.md}{4.5 Выполнение "make clean" с помощью catkin}} 
\item { \href{4.6-vosstanovlenie-single-catkin-package.md}{4.6 Восстановление Single catkin Package}} 
\item { \href{4.7-smeshivanie-rabochikh-prostranstv-catkin-i-rosbuild.md}{4.7 Смешивание рабочих пространств catkin и rosbuild}} 
\item { \href{4.8-rabota-s-official-ros-tutorials.md}{4.8 Работа с Official ROS Tutorials}} 
\item { \href{4.9-rviz-instrument-vizualizacii-ros.md}{4.9 RViz: инструмент визуализации ROS}} 
\item { \href{4.10-ispolzovanie-parametrov-ros-v-vashikh-programmakh.md}{4.10 Использование параметров ROS в ваших программах}} 
\item { \href{4.11-ispolzovanie-rqt_reconfigure-dlya-ustanovki-parametrov-ros.md}{4.11 Использование rqt\_reconfigure для установки параметров ROS}} 
\item { \href{4.12-set-mezhdu-robotom-i-kompyuterom.md}{4.12 Сеть между роботом и компьютером}} 
\item { \href{4.12.1-sinkhronizaciya-vremeni.md}{4.12.1 Синхронизация Времени}} 
\item { \href{4.12.2-set-ros-s-ispolzovaniem-zeroconf.md}{4.12.2 Сеть ROS с использованием Zeroconf}} 
\item { \href{4.12.3-testirovanie-podklyucheniya.md}{4.12.3 Тестирование Подключения}} 
\item { \href{4.12.4-ustanovka-peremennykh-ros_master_uri-i-ros_hostname.md}{4.12.4 Установка переменных ROS\_MASTER\_URI и ROS\_HOSTNAME}} 
\item { \href{4.12.5-otkrytie-novykh-terminalov.md}{4.12.5 Открытие Новых Терминалов}} 
\item { \href{4.12.6-running-nodes-on-both-machines.md}{4.12.6 Running Nodes on both Machines}} 
\item { \href{4.12.7-set-ros-cherez-internet.md}{4.12.7 Сеть ROS через Интернет}} 
\item { \href{untitled-7.md}{4.13 ROS повторение}} 
\item { \href{untitled-8.md}{4.14 что такое приложение ROS?}} 
\item { \href{untitled-9.md}{4.15 Установка пакетов с помощью SVN, Git и Mercurial}} 
\item { \href{4.15.1-svn.md}{4.15.1 SVN}} 
\item { \href{4.15.2-git.md}{4.15.2 Git}} 
\item { \href{4.15.3-mercurial.md}{4.15.3 Mercurial}} 
\item { \href{untitled-10.md}{4.16 Удаление пакетов из Personal catkin Directory}} 
\item { \href{untitled-11.md}{4.17 Как найти сторонние ROS-пакеты}} 
\item { \href{4.17.1-poisk-v-ros-wiki-ros-wiki.md}{4.17.1 Поиск в ROS Wiki ROS Wiki}} 
\item { \href{4.17.2-ispolzovanie-komandy-roslocate.md}{4.17.2 Использование команды roslocate}} 
\item { \href{4.17.3-prosmotr-indeksa-programmnogo-obespecheniya-ros.md}{4.17.3 Просмотр индекса программного обеспечения ROS}} 
\item { \href{4.17.4.-poisk-v-google.md}{4.17.4. Поиск в Google}} 
\item { \href{untitled.md}{4.18 Получение дальнейшей помощи с ROS}} 
\item { \href{untitled-1.md}{5. УСТАНОВКА КОДА ROS НА ПРИМЕРЕ}} 
\item { \href{untitled-2.md}{5.2 Клонирование репозитория Indigo Ros-by-example}} 
\item { \href{5.2.1-obnovlenie-ot-electric-ili-fuerte.md}{5.2.1 Обновление от Electric или Fuerte}} 
\item { \href{5.2.2-obnovlenie-s-groovy.md}{5.2.2 Обновление с Groovy}} 
\item { \href{5.2.3-obnovlenie-s-hydro.md}{5.2.3 Обновление с Hydro}} 
\item { \href{5.2.4.-klonirovanie-repozitoriya-rbx1-dlya-indigo-v-pervyi-raz.md}{5.2.4. Клонирование репозитория rbx1 для Indigo в первый раз}} 
\item { \href{5.3-o-spiskakh-kodov-v-etoi-knige.md}{5.3 О списках кодов в этой книге}} 
\item { \href{6.-ustanovka-arbotix-simulator.md}{6. УСТАНОВКА ARBOTIX SIMULATOR}} 
\item { \href{6.1-ustanovka-simulyatora.md}{6.1 Установка симулятора}} 
\item { \href{6.2-testirovanie-simulyatora.md}{6.2 Тестирование симулятора}} 
\item { \href{6.3-zapusk-simulyatora-s-vashim-sobstvennym-robotom.md}{6.3 Запуск симулятора с вашим собственным роботом}} 
\item { \href{7.-kontrol-mobilnoi-bazy.md}{7. КОНТРОЛЬ МОБИЛЬНОЙ БАЗЫ}} 
\item { \href{7.1-edinicy-i-sistemy-koordinat.md}{7.1 Единицы и системы координат}} 
\item { \href{7.2-urovni-upravleniya-dvizheniem.md}{7.2 Уровни управления движением}} 
\item { \href{7.3-skruchivanie-i-povorot-s-pomoshyu-ros.md}{7.3 Скручивание и поворот с помощью ROS}} 
\item { \href{untitled-17.md}{7.4 Калибровка одометрии вашего робота}} 
\item { \href{untitled-18.md}{7.5 Отправка твист-сообщений реальному роботу}} 
\item { \href{untitled-19.md}{7.6 Публикация твист-сообщений от узла ROS}} 
\item { \href{untitled-3.md}{7.7 Мы Уже Там? Подойдя на расстояние с одометра}} 
\item { \href{untitled-4.md}{7.8 Туда и обратно, используя одометрию}} 
\item { \href{untitled-5.md}{7.9 Навигация по квадрату с помощью одометрии}} 
\item { \href{untitled-6.md}{7.10 Телеоперация вашего робота}} 
\item { \href{9.-raspoznavanie-i-sintez-rechi.md}{9. РАСПОЗНАВАНИЕ И СИНТЕЗ РЕЧИ}} 
\item { \href{9.1-ustanovka-pocketsphinx-dlya-raspoznavaniya-rechi.md}{9.1 Установка PocketSphinx для распознавания речи}} 
\item { \href{9.2.-testirovanie-raspoznavatelya-pocketsphinx.md}{9.2. Тестирование распознавателя PocketSphinx}} 
\item { \href{9.3-sozdanie-slovarnogo-zapasa.md}{9.3 Создание словарного запаса}} 
\item { \href{9.4-skript-navigacii-s-golosovym-upravleniem.md}{9.4 Скрипт навигации с голосовым управлением}} 
\item { \href{9.4.1.-testirovanie-golosovogo-upravleniya-v-simulyatore-arbotix.md}{9.4.1. Тестирование голосового управления в симуляторе ArbotiX}} 
\item { \href{9.4.2.-ispolzovanie-golosovogo-upravleniya-s-nastoyashim-robotom.md}{9.4.2. Использование голосового управления с настоящим роботом}} 
\item { \href{9.5-ustanovka-i-testirovanie-festival-text-to-speech.md}{9.5 Установка и тестирование Festival Text-to-Speech}} 
\item { \href{9.5.1.-ispolzovanie-preobrazovaniya-teksta-v-rech-v-uzle-ros.md}{9.5.1. Использование преобразования текста в речь в узле ROS}} 
\item { \href{9.5.2.-testirovanie-scenariya-talkback.py.md}{9.5.2. Тестирование сценария talkback.py}} 
\item { \href{10.-zrenie-robota.md}{10. ЗРЕНИЕ РОБОТА}} 
\item { \href{10.1-opencv-openni-and-pcl.md}{10.1 OpenCV, OpenNI and PCL}} 
\item { \href{10.2-primechanie-o-razreshenii-kamery.md}{10.2 Примечание о разрешении камеры}} 
\item { \href{10.3-ustanovka-i-testirovanie-draiverov-kamery-ros.md}{10.3 Установка и тестирование драйверов камеры ROS}} 
\item { \href{10.4-ustanovka-opencv-v-ubuntu-linux.md}{10.4 Установка OpenCV в Ubuntu Linux}} 
\item { \href{10.5-ros-v-opencv-paket-cv_bridge.md}{10.5 ROS в OpenCV: пакет cv\_bridge}} 
\item { \href{10.6-the-ros2opencv2.pyutility.md}{10.6 The ros2opencv2.pyUtility}} 
\item { \href{10.7-obrabotka-zapisannogo-video.md}{10.7 Обработка записанного видео}} 
\item { \href{10.8-opencv-biblioteka-kompyuternogo-zreniya-s-otkrytym-iskhodnym-kodom.md}{10.8 OpenCV: библиотека компьютерного зрения с открытым исходным кодом}} 
\item { \href{10.9-openni-i-otslezhivanie-skeleta.md}{10.9 OpenNI и отслеживание скелета}} 
\item { \href{10.10-uzly-pcl-i-trekhmernye-oblaka-tochek.md}{10.10 Узлы PCL и трехмерные облака точек}} 
\end{itemize} 

