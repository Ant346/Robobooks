# 10.2 Примечание о разрешении камеры


Большинство видеокамер, используемых в робототехнике, могут работать с различными разрешениями, обычно 160x120 (QQVGA), 320x240 (QVGA), 640x480 (VGA) и иногда до 1280x960 (SXGA). Часто соблазнительно установить камеру на максимально возможное разрешение при условии, что больше деталей будет лучше. Однако большее количество пикселей достигается за счет большего количества вычислений на кадр. Например, каждый кадр видео 1280x960 содержит в четыре раза больше пикселей, чем видео 640x480. Это означает в четыре раза больше операций на кадр для большинства видов базовой обработки изображений. Если ваш процессор и графический процессор уже разогнаны при обработке видео 640x480 со скоростью 20 кадров в секунду, вам повезет получить 5 кадров в секунду для видео 1280x960, которое часто слишком медленное для обработки изображения в реальном времени на мобильном устройстве. робот. Кроме того, все эти дополнительные пиксели редко нужны для базовых задач зрения, таких как отслеживание цвета или распознавание лица.

Файлы запуска камеры по умолчанию, распространяемые с ROS, используют разрешение 640x480. Если вам нужно более низкое или более высокое разрешение, вы можете использовать rqt\_reconfigure или файл запуска камеры, чтобы изменить режимы разрешения, как мы проиллюстрируем ниже.

\textbf{ПРИМЕЧАНИЕ}. Вспомните из более ранней книги, что при использовании Microsoft Kinect и драйвера freenect 640x480 (VGA) является самым низким доступным разрешением.

