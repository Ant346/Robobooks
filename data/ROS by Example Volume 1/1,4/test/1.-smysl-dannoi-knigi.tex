\Section{1. СМЫСЛ ДАННОЙ КНИГИ}


ROS чрезвычайно мощен и продолжает расширяться и улучшаться быстрыми темпами. И все же одна из проблем, с которой сталкиваются многие новые пользователи ROS, - это знать, с чего начать. На самом деле есть два этапа для начала работы с ROS: этап 1 включает изучение основных понятий и методов программирования, а этап 2 посвящен использованию ROS для управления вашим собственным роботом.

Этап 1 лучше всего решать, обратившись к \href{http://wiki.ros.org/}{ROS Wiki}, где вы найдете набор \href{http://wiki.ros.org/indigo/Installation}{инструкций} по установке и коллекцию великолепно написанных \href{http://wiki.ros.org/ROS/Tutorials}{учебников} для начинающих. Эти учебники были проверены в бою сотнями, если не тысячами пользователей, поэтому нет смысла дублировать их здесь. Эти учебные пособия считаются обязательным условием для использования этой книги. Поэтому мы будем считать, что читатель хотя бы раз проработал все руководства. Также важно прочитать обзор \href{http://wiki.ros.org/tf}{tf overview} и выполнить \href{http://wiki.ros.org/tf/Tutorials}{tf Tutorials}, которые помогут вам понять, как ROS работает с различными системами отсчета. Если у вас возникли проблемы, посетите форум ответов \href{http://answers.ros.org/}{ROS Answers}(\href{http://answers.ros.org/}{http://answers.ros.org}), на который, возможно, уже есть ответы на многие ваши вопросы. Если нет, вы можете опубликовать свой новый вопрос там. (Пожалуйста, не используйте список рассылки ros-users для таких вопросов, который зарезервирован для новостей и объявлений ROS.)

Фаза 2-это то, о чем вся эта книга: использование ROS, чтобы заставить вашего робота выполнять некоторые довольно впечатляющие задачи. В каждой главе будут представлены учебные пособия и примеры кода, связанные с различными аспектами ROS. Затем мы применим этот код либо к реальному роботу, либо к наклонной головке, либо даже просто к камере (например, распознавание лиц). По большей части вы можете делать эти уроки в любом порядке, который вам нравится. В то же время учебники будут строиться друг на друге, так что к концу книги ваш робот сможет автономно перемещаться по вашему дому или офису, реагировать на произносимые команды и комбинировать управление зрением и движением, чтобы отслеживать лица или следовать за человеком по дому.



В этом томе мы рассмотрим следующие темы:

\begin{itemize} 
\item { Установка и настройка ROS (обзор).} 
\item { Управление мобильной базой на разных уровнях абстракции, начиная с водителей мотора и колесных кодировщиков и заканчивая планированием пути и составлением карты.} 
\item { Навигация и SLAM (одновременная локализация и картирование) с использованием лазерного сканера или глубокой камеры, такой как Microsoft Kinect или Asus Xtion.} 
\item { Распознавание и синтез речи, а также приложение для управления вашим роботом с помощью голосовых команд.} 
\item { Распознавание и синтез речи, а также приложение для управления вашим роботом с помощью голосовых команд.} 
\item { Объединение зрения робота с мобильной базой для создания двух приложений, одно из которых предназначено для отслеживания лиц и цветных объектов, а другое-для слежения за человеком, когда он перемещается по комнате.} 
\item { Управление камерой панорамирования и наклона с помощью сервоприводов Dynamixel для отслеживания движущегося объекта.} 
\item { Зрение робота, включая распознавание лиц и отслеживание цвета с помощью OpenCV, отслеживание скелета с помощью OpenNI и краткое введение в PCL для обработки 3D-зрения.} 
\item { Объединение зрения робота с мобильной базой для создания двух приложений, одно из которых предназначено для отслеживания лиц и цветных объектов, а другое-для слежения за человеком, когда он перемещается по комнате.} 
\item { Управление камерой панорамирования и наклона с помощью сервоприводов Dynamixel для отслеживания движущегося объекта.} 
\end{itemize} 

Учитывая широту и глубину рамок ROS, мы обязательно должны опустить несколько тем в этом вводном томе. В частности, следующие темы не рассматриваются до\href{http://moveit.ros.org/}{ Тома 2}: симулятор \href{http://moveit.ros.org/}{Gazebo}, создание собственной модели робота \href{http://wiki.ros.org/urdf/Tutorials}{URDF}, управление многосуставчатым рычагом и захватом с помощью \href{http://moveit.ros.org/}{MoveIt!} (ранее \href{%20http://wiki.ros.org/arm_navigation}{ARM navigation}), \href{http://wiki.ros.org/diagnostics}{диагностика роботов}, использование менеджеров задач, таких как\href{http://wiki.ros.org/executive_smach}{ SMACH} и \href{http://wiki.ros.org/rosbridge_suite}{rosbridge}, для создания веб-приложений ROS. Тем не менее, как вы можете видеть из приведенного выше списка, нам еще многое предстоит сделать!




