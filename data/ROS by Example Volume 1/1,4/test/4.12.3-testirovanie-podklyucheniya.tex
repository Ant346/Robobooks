# 4.12.3 Тестирование Подключения




Используйте команду ping, чтобы убедиться, что у вас есть базовое подключение между двумя компьютерами. С вашего компьютера выполните команду:

```text
$ ping my_robot.local
```

что должно привести к следующему результату:

> PING my\_robot.local (192.168.0.197) 56(84) bytes of data.  
>  64 bytes from my\_robot.local (192.168.0.197): icmp\_req=1 ttl=64 time=1.65 ms  
>  64 bytes from my\_robot.local (192.168.0.197): icmp\_req=2 ttl=64 time=0.752 ms  
>  64 bytes from my\_robot.local (192.168.0.197): icmp\_req=3 ttl=64 time=1.69 ms

Введите Ctrl-C, чтобы остановить тест. Переменная icmp\_req подсчитывает пинги, в то время как переменная time указывает время обратного хода в миллисекундах. После остановки теста вы получите резюме, которое выглядит следующим образом:

> ---my\_robot.local ping statistics---  
>  3 packets transmitted, 3 received, 0% packet loss, time 2001ms rtt min/avg/max/mdev = 0.752/1.367/1.696/0.436 ms

В общем случае вы должны видеть потерю пакетов 0% и среднее время задержки около 5 мс.

 Теперь сделайте тест в другом направлении. Вызовите терминал на вашем роботе (или используйте ssh, если вы знаете, что он уже работает) и пропингуйте свой рабочий стол:

```text
$ ping my_desktop.local
```

Еще раз вы должны увидеть 0% потери пакетов и короткое время поездки туда и обратно. 

\textbf{Примечание}: Если тест ping не проходит с ошибкой "неизвестный хост", попробуйте перезапустить avahi- процесс демона на машине, которая не отвечает на запросы:

```text
$ sudo service avahi-daemon restart
```

