\Subsection{4.3 Построение пакетов ROS с помощью Catkin}


В вики ROS существует несколько пошаговых руководств по \href{http://wiki.ros.org/catkin/Tutorials}{catkin tutorials}, и читателю предлагается пройти хотя бы первые четыре. В этой главе мы приведем только краткое изложение основных моментов.

Если вы следовали инструкциям по установке Ubuntu ROS, все пакеты и метапакеты ROS будут находиться в / opt / ros / release, где release - это имя выпуска ROS, который вы используете; например / Опт / ROS / индиго. Это часть файловой системы, предназначенная только для чтения, и ее не следует изменять, кроме как через менеджер пакетов, поэтому вы захотите создать личный каталог ROS в своем домашнем каталоге, чтобы вы могли (1) установить сторонние пакеты ROS, которые не иметь версий Debian и (2) создавать свои собственные пакеты ROS.

