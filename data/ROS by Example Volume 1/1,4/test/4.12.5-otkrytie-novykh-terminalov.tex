# 4.12.5 Открытие Новых Терминалов


В любом новом окне терминала, которое вы открываете на своем рабочем столе или роботе, вам нужно установить переменную ROS\_HOSTNAME на имя Zeroconf этой машины. А для новых настольных терминалов вы также должны установить ROS\_MASTER\_URI, чтобы он указывал на робота. (Или в более общем смысле, на любом компьютере, не являющемся главным, новые терминалы должны установить ROS\_MASTER\_URI для указания на главный компьютер.)

Если вы будете использовать один и тот же робот и рабочий стол в течение некоторого времени, вы можете сэкономить некоторое время, добавив соответствующие строки экспорта в конец ~/.файл bashrc на каждом компьютере. Если робот всегда будет хозяином, добавьте следующую строку в конец его ~/.файл bashrc:

> export ROS\_HOSTNAME=my\_robot.local

А на рабочем столе добавьте следующие две строки в конец его ~/.файл bashrc:

> export ROS\_HOSTNAME=my\_desktop.local
>
> export ROS\_MASTER\_URI=http://my\_robot.local:11311

(Конечно, замените имена Zeroconf теми, которые соответствуют вашей настройке.) 

Вы также можете настроить свой рабочий стол в качестве мастера ROS вместо робота. В этом случае просто измените роли и имя хоста Zeroconf в приведенных выше примерах.

