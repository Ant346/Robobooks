\Subsection{4.7 Смешивание рабочих пространств catkin и rosbuild}


\textbf{Примечание}: Больше не рекомендуется смешивать пакеты rosbuild и catkin. Если у вас все еще есть более старые пакеты rosbuild, которые не имеют эквивалента в catkin, рассмотрите возможность \href{http://wiki.ros.org/catkin/Tutorials/convert_rosbuild_to_catkin}{миграции} этих пакетов в catkin. Остальная часть этого раздела оставлена ​​здесь от Гидро-пересмотра книги по старым причинам. Он будет удален в следующей редакции книги.

Если вы некоторое время использовали ROS, у вас, вероятно, уже есть рабочее пространство ROS и пакеты, которые используют более раннюю систему rosbuild make, а не catkin. Вы можете продолжать использовать эти пакеты и rosmake, продолжая использовать catkin для новых пакетов.

Предполагая, что вы выполнили действия, описанные в предыдущем разделе, и что каталогом рабочей области rosbuild является ~ / ros\_workspace, выполните следующую команду, чтобы позволить двум системам работать вместе:

```text
$ rosws init ~/ros_workspace ~/catkin_ws/devel
```

Конечно, измените имена каталогов в приведенной выше команде, если вы создали свои рабочие пространства rosbuild и/или catkin в других местах. Примечание: Если вы получаете следующую ошибку при выполнении приведенной выше команды:

```text
rosws: command not found
```

Это означает, что вы не установили файлы rosinstall во время первоначальной установки ROS. (Это последний шаг в \href{http://wiki.ros.org/indigo/Installation/Ubuntu%23Getting_rosinstall}{руководстве по установке}.) Если это так, установите rosinstall прямо сейчас:

```text
$ sudo apt-get install python-rosinstall
```

И снова попробуйте выполнить команду rosws. После завершения этого шага отредактируйте свой ~/.bashrc файл и изменить строку, которая выглядит следующим образом:

```text
source /opt/ros/indigo/setup.bash
```

затем:

```text
source ~/ros_workspace/setup.bash
```

Опять же, при необходимости измените имя каталога для рабочей области rosbuild. Сохраните изменения в ~/.bashrc и выйдите из своего редактора. Чтобы сделать новую объединенную рабочую область активной немедленно, выполните команду:

```text
$ source ~/ros_workspace/setup.bash
```

Новые окна терминала автоматически выполнят эту команду из вашего ~/.файл bashrc.

