\Subsection{3.2 Начало работы с Linux}


Если вы уже являетесь опытным пользователем Linux, вы значительно опередили игру. Если нет, возможно, вы захотите пройти учебник или два, прежде чем продолжить. Поскольку веб-учебники приходят и уходят каждый день, вы можете просто найти в Google что-то вроде «Учебник по Ubuntu» и найти то, что вам нравится. Однако имейте в виду, что большая часть вашей работы с ROS будет выполняться в командной строке или в текстовом редакторе. Хорошее место для начала работы с основами командной строки - использование \href{https://help.ubuntu.com/community/UsingTheTerminal}{Ubuntu Terminal}. Текстовый редактор, который вы выбираете для программирования, зависит от вас. Выбор включает в себя gedit, nano, pico, emacs, vim, Eclipse и многие другие. (См., Например, \href{https://help.ubuntu.com/community/Programming}{https://help.ubuntu.com/community/Programming}.) Такие программы, как Eclipse, на самом деле являются полнофункциональными IDE и могут использоваться для организации проектов, тестирования кода, управления репозиториями SVN и Git и так далее. Для получения дополнительной информации об использовании различных IDE с \href{http://wiki.ros.org/IDEs}{ROS см. Http://wiki.ros.org/IDEs}.

