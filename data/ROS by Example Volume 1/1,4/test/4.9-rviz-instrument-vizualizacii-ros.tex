\Subsection{4.9 RViz: инструмент визуализации ROS}


Если вы уже некоторое время работаете с ROS, то, вероятно, знакомы с \href{http://wiki.ros.org/rviz}{RViz}, универсальной программой визуализации ROS. Однако, поскольку RViz не охватывается стандартными учебными пособиями для начинающих ROS, вы, возможно, не были официально представлены. К счастью, уже существует \href{http://ros.org/doc/indigo/api/rviz/html/user_guide/}{руководство пользователя RViz}, которое проведет вас шаг за шагом через его функции. Пожалуйста, не забудьте прочитать руководство, прежде чем идти дальше, так как мы будем интенсивно использовать RViz в нескольких частях книги.

RViz иногда может быть немного привередлив в работе на разных графических картах. Если вы обнаружите, что RViz прерывается во время запуска, сначала просто попробуйте запустить его снова. Продолжайте пробовать несколько раз, если это необходимо. (На одном из моих компьютеров обычно срабатывает третья попытка.) Если это не удается, ознакомьтесь с \href{http://ros.org/wiki/rviz/Troubleshooting}{руководством по устранению неполадок RViz} для возможных решений. В частности, раздел о \href{http://ros.org/wiki/rviz/Troubleshooting%23Segfault_during_startup}{Segfaults} во время запуска имеет тенденцию решать наиболее распространенные проблемы. Если ваша видеокарта не поддерживает аппаратное ускорение OpenGL, выключите ее, следуя приведенным \href{http://wiki.ros.org/rviz/Troubleshooting%23Turning_off_hardware_acceleration}{здесь} инструкциям.

