\Subsection{3.3 Примечание об обновлениях и изменениях}


У крупных разработчиков программного обеспечения наблюдается тенденция к сокращению цикла выпуска. Некоторые пакеты, такие как Firefox, сейчас работают 6-недельным циклом.

Также кажется, что в этих циклах быстрого обновления наблюдается растущая тенденция к нарушению работы кода, который отлично работал днем ​​ранее. Много времени можно потратить на погоню за последним выпуском, а затем на исправление всего кода, чтобы он снова заработал. Как правило, перед обновлением рекомендуется проверить список изменений для определенного продукта и убедиться, что он действительно вам нужен. В противном случае, если он не сломался ...

Недавний опрос разработчиков и пользователей ROS показал, что желателен более длительный цикл поддержки, и разработчики ядра ROS из OSRF с благодарностью создали ROS Indigo в качестве своего первого выпуска LTS, соответствующего графику выпуска Ubuntu LTS. Вы можете найти подробности на \href{http://wiki.ros.org/Distributions}{странице} «Распространения» в ROS Wiki и в расписании \href{https://wiki.ubuntu.com/Releases}{релизов} Ubuntu. В частности, использование ROS Indigo в Ubuntu Trusty (14.04) будет поддерживаться до апреля 2019 года.

