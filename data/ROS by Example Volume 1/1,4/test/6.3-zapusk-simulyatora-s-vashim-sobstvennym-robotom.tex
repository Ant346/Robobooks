\Subsection{6.3 Запуск симулятора с вашим собственным роботом}


Если у вас есть модель \href{%20http://wiki.ros.org/urdf/Tutorials}{URDF} вашего собственного робота, вы можете запустить ее в симуляторе ArbotiX вместо TurtleBot или Pi Robot. Сначала сделайте копию поддельного файла запуска TurtleBot:

```text
$ roscd rbx1_bringup/launch
$ cp fake_turtlebot.launch fake_my_robot.launch
```

Затем откройте файл запуска в вашем любимом редакторе. Сначала это будет выглядеть так:

> &lt;launch&gt;  
>  &lt;param name="/use\_sim\_time" value="false" /&gt;
>
> &lt;!-- Load the URDF/Xacro model of our robot --&gt;
>
> &lt;arg name="urdf\_file" default="$(find xacro)/xacro.py '$(find rbx1\_description)/urdf/turtlebot.urdf.xacro'" /&gt;
>
> &lt;param name="robot\_description" command="$(arg urdf\_file)" /&gt;
>
> &lt;node name="arbotix" pkg="arbotix\_python" type="arbotix\_driver" output="screen"&gt;
>
> &lt;rosparam file="$(find rbx1\_bringup)/config/fake\_turtlebot\_arbotix.yaml" command="load" /&gt;
>
> &lt;param name="sim" value="true"/&gt; &lt;/node&gt;
>
> &lt;node name="robot\_state\_publisher" pkg="robot\_state\_publisher" type="state\_publisher"&gt;
>
> &lt;param name="publish\_frequency" type="double" value="20.0" /&gt; &lt;/node&gt;
>
> &lt;/launch&gt;

Как вы можете видеть, модель URDF загружается в верхней части файла. Просто замените имена пакетов и путей, чтобы они указывали на ваш собственный файл URDF / Xacro. Вы можете оставить большинство остальных файлов запуска одинаковыми. Результат будет выглядеть примерно так:

> &lt;launch&gt;  
>  &lt;!-- Load the URDF/Xacro model of our robot --&gt;  
>  &lt;arg name="urdf\_file" default="$(find xacro)/xacro.py '$(find
>
> YOUR\_PACKAGE\_NAME)/YOUR\_URDF\_PATH'" /&gt;
>
> &lt;param name="robot\_description" command="$(arg urdf\_file)" /&gt;
>
> &lt;node name="arbotix" pkg="arbotix\_python" type="arbotix\_driver" output="screen"&gt;
>
> &lt;rosparam file="$(find rbx1\_bringup)/config/fake\_turtlebot\_arbotix.yaml" command="load" /&gt;
>
> &lt;param name="sim" value="true"/&gt; &lt;/node&gt;
>
> &lt;node name="robot\_state\_publisher" pkg="robot\_state\_publisher" type="state\_publisher"&gt;
>
> &lt;param name="publish\_frequency" type="double" value="20.0" /&gt; &lt;/node&gt;
>
> &lt;/launch&gt;

Если ваш робот имеет руку или голова панорамирования и наклона, вы можете начать с файла fake\_pi\_robot.launch в качестве шаблона

